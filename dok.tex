\documentclass[a4paper,12pt]{article}
\usepackage{amsmath}
\usepackage{amssymb}
\usepackage[polish]{babel}
\usepackage{polski}
\usepackage[utf8]{inputenc}
\usepackage{indentfirst}
\usepackage{geometry}
\usepackage[font={small,it}]{caption}
\usepackage{array}
\usepackage[pdftex]{color,graphicx}
\usepackage{subfigure}
\usepackage{afterpage}
\usepackage{setspace}
\usepackage{color}
\usepackage{wrapfig}
\usepackage{listings}
\usepackage{datetime}

\renewcommand{\onehalfspacing}{\setstretch{1.6}}

\geometry{tmargin=2.5cm,bmargin=2.5cm,lmargin=2.5cm,rmargin=2.5cm}
\setlength{\parindent}{1cm}
\setlength{\parskip}{0mm}

\newenvironment{lista}{
\begin{itemize}
  \setlength{\itemsep}{1pt}
  \setlength{\parskip}{0pt}
  \setlength{\parsep}{0pt}
}{\end{itemize}}

\newcommand{\linia}{\rule{\linewidth}{0.4mm}}

\definecolor{lbcolor}{rgb}{0.95,0.95,0.95}
\lstset{
    backgroundcolor=\color{lbcolor},
    tabsize=4,
  language=C++,
  captionpos=b,
  tabsize=3,
  frame=lines,
  numbers=left,
  numberstyle=\tiny,
  numbersep=5pt,
  breaklines=true,
  showstringspaces=false,
  basicstyle=\footnotesize,
  identifierstyle=\color{magenta},
  keywordstyle=\color[rgb]{0,0,1},
  commentstyle=\color{Darkgreen},
  stringstyle=\color{red}
  }

\begin{document}

\noindent
\begin{tabular}{|c|p{11cm}|c|} \hline 
Zofia Sobocińska & \ddmmyyyydate\today \tabularnewline
\hline 
\end{tabular}


\section*{Zadanie 5 - Mnożenie macierzy na kartach graficznych}

Program implementuje mnożenie macierzy $ A_{N \times N} \times B_{N \times N} = C_{N \times N} $ w technologii CUDA. Zadanie jest podzielone na bloki o rozmiarze $4 \times 4$. Takich bloków wykorzystanych jest $floor(\frac{N + 4}{4} - 1) \times floor(\frac{N + 4}{4} - 1)$.

\begin{lstlisting}
__global__ void multiple(float* C, float* A, float* B, uint size) {

    float c = 0.0;

    uint yc = blockIdx.y * BLOCK_SIZE + threadIdx.y;
    uint xc = blockIdx.x * BLOCK_SIZE + threadIdx.x;

    uint GRID_SIZE = (BLOCK_SIZE + size - 1) / BLOCK_SIZE;

    if (xc < size && yc < size) {

        for (uint i = 0; i < GRID_SIZE; i++) {
            for (uint j = 0; j < BLOCK_SIZE; ++j) {
                uint index = i * BLOCK_SIZE + j;
                if ((index < size) && (index < size)) {
                    c += A[yc * size + index] * B[index * size + xc];
                }
            }
        }
        C[yc * size + xc] = c;
    }
}
\end{lstlisting}

\subsection*{Czas obliczeń}

Średnie czasy obliczeń mnożenia macierzy $128 \times 128$ (przy całkowitej liczbie iteracji $100000$) w technologii CUDA i na CPU (wielo- bądź jednowątkowo) podane zostały w tabeli:

\begin{table}[!hbp]
\centering
\begin{tabular}{|cl|c|c|}
\hline 
& & średni czas & przyśpieszenie \\
& środowisko & obliczeń & względem 1 wątku\\
\hline 
CPU & sekwencyjnie & $9.911ms$ &\\
& OpenMP - 4 wątki & $3.282ms$ & $3 \times$ \\
GPU & CUDA & $0.570 ms$ & $17 \times$ \tabularnewline
\hline
\end{tabular}
\caption{Porównanie czasów obliczeń mnożenia macierzy w różnych technologiach}
\end{table}

\subsection*{Podsumowanie}

Technologia CUDA wymaga od programisty zmiany koncepcji myślenia o projektowaniu algorytmów. Wewnątrz kernela - kodu wątku, który ma się wykonać na GPU - dysponujemy nie numerem iteracji, ale identyfikatorem bloku i wątku, na podstawie których określamy, co chcemy zrobić z danymi. Dodatkowo możemy przekazać kernelowi wskaźnik na obszar pamięci zaalokowany w RAMie karty graficznej, na który (z którego) możemy możliwość kopiować nasze dane z (do) pamięci zaalokowanej w RAMie hosta.

\subsection*{Od autora}

Pragnę przeprosić za spóźniony czas oddania sprawozdania. Przyczyną tegoż stał się błąd z gatunku kardynalnych, przez który wypisywany czas obliczeń pozostawał stale na poziomie 0ms. W celu uniknięcia podobnych błędów kardynalnych w przyszłości pozostanę przy głupotoodpornym przekierowywaniu strumienia do std::cout.

\end{document}
